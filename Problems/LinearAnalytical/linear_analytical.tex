% Six results 2 - Excellent, 2 Good, 2 Bad. For softplus show halos
% Mention adding more noise
% Use only active voice. 
\documentclass[12pt]{article}
\usepackage{amsmath}
\usepackage{amsfonts}
% to break urls: https://tex.stackexchange.com/questions/115690/urls-in-bibliography-latex-not-breaking-line-as-expected
\newcommand{\beq}{\begin{equation}}
\newcommand{\eeq}{\end{equation}}
\newcommand{\ber}{\begin{eqnarray}}
\newcommand{\eer}{\end{eqnarray}}
\newcommand{\dd}[2]{\frac{d}{d{#2}}{(#1)} }
\newcommand{\pdd}[2]{\frac{\partial{{#1}}}{\partial{#2}}}
\begin{document}
% bib
\title{Analytical solution for linear elasticity}
\date{\today}
\author{Nachiket Gokhale}
\maketitle
\section{Axial problem}
Consider a 3D rectangular elastic domain, infinite in the z-direction thus yielding a plane strain problem. The domain is  $[0,L_x]\times[0,L_y]\times(-\infty,\infty)$, with traction free sides, traction specified on the top surface i.e. $\sigma\cdot{\mathbf{n}}=(0,h_y,0)$ on top surface, $u_x(0,0)=0,u_y(0,0)=0$ and $u_y(0,L_x)=0$. Assume a solution
\beq
u_x=ax \qquad u_y = by \qquad u_z = 0\,\, \text{plane strain}
\eeq
This gives
\beq
\epsilon = \begin{pmatrix}a & 0 & 0 \\ 0 & b & 0 \\ 0 & 0 & 0 \end{pmatrix}
\eeq
and,
\ber
\sigma &=& \lambda\text{tr}(\epsilon)\begin{pmatrix}1 & 0 & 0\\ 0 & 1 & 0\\0 & 0 & 1\end{pmatrix} + 2\mu\epsilon \\
  &=& \lambda(a+b)\begin{pmatrix}1 & 0 & 0\\ 0 & 1 & 0\\0 & 0 & 1\end{pmatrix} +  2\mu\begin{pmatrix}a & 0 & 0 \\ 0 & b & 0 \\ 0 & 0 & 0 \end{pmatrix} \\
  &=& \begin{pmatrix}\lambda(a+b)+2\mu{a} & 0 & 0 \\ 0 & \lambda(a+b)+2\mu{b} &0 \\ 0 & 0 & \lambda(a+b)\end{pmatrix}  \\
  &=& \begin{pmatrix} (\lambda+2\mu)a + \lambda{b}& 0 & 0\\0 & \lambda{a} + (\lambda+2\mu)b &0 \\ 0 & 0 & \lambda(a+b) \end{pmatrix} 
      \eer
      
      
% spurious indentation going on here
 Traction free sides implies
      \ber
      \sigma\cdot\begin{pmatrix}1 \\ 0 \\ 0\end{pmatrix} &=& 0 \implies      (\lambda+2\mu)a + \lambda{b}=0 \\
      a &=& -\frac{\lambda}{(\lambda+2\mu)}b \label{eqn:defa}
      \eer

      Traction condition on the top
      \ber
      \sigma\cdot\begin{pmatrix}0\\1\\0\end{pmatrix} = \begin{pmatrix}0\\h_y\\0\end{pmatrix} \implies \lambda{a} + (\lambda+2\mu)b = h_y
        \eer

        using equation (\ref{eqn:defa}) in the above equation

        \ber
        \frac{-\lambda^2b}{(\lambda+2\mu)} + (\lambda+2\mu)b &=& h_y \\
        (-\lambda^2 + (\lambda+2\mu)^2)b &=& h_y(\lambda+2\mu)\\
        (-\lambda^2 + \lambda^2 + 4\lambda\mu + 4\mu^2)b &=& h_y(\lambda+2\mu)\\
        4\mu(\lambda+\mu)b &=& h_y(\lambda+2\mu)
        b = h_y\frac{(\lambda+2\mu)}{4\mu(\lambda+\mu)}
        \eer

        To summarize
        \beq
        b = h_y\frac{(\lambda+2\mu)}{4\mu(\lambda+\mu)} \qquad  a = -\frac{\lambda}{(\lambda+2\mu)}b 
        \eeq
        %
Note: Plane strain is \textit{not} uniaxial tension. There is axial stress in the infinite direction.      
\section{Shear problem}        
Consider a 3D rectangular elastic domain, infinite in the z-direction thus yielding a plane strain problem. The domain is  $[0,L_x]\times[0,L_y]\times(-\infty,\infty)$. Assume the displacement field in the domain to be
\beq
u_x = ay \qquad u_y = bx \qquad u_z = 0
\eeq
The strain tensor and stress tensor are
\ber
\epsilon = \begin{pmatrix}0 & 0.5(a+b) & 0\\ 0.5(a+b) & 0 & 0 \\ 0 & 0 & 0\end{pmatrix}\\  \, \sigma = 2\mu\epsilon = \begin{pmatrix}0 & \mu(a+b) & 0\\ \mu(a+b) & 0 & 0 \\ 0 & 0 & 0\end{pmatrix}  
\eer
Since stress tensor is a constant, the equations of equilibrium are satisfied. We can use this stress tensor to compute what the traction should be on the faces, if one wanted to impose traction boundary conditions. 
\end{document}
